%&latex
\section{Main Event Loop}

The main  event loop  in both \textsc{wmRS}  and \textsc{wmCPP}  comprises three
core  stages. They  first  rely on  the underlying  windowing  system to  report
certain \textit{events} they  are interested in. This is  done synchronously, as
without any hardware interrupts or changes  to the environment, nothing is to be
done.  That  is, both  window  managers  \textit{block}  until  a new  event  is
received. The second  stage is the \textit{extraction}  and \textit{bundling} of
useful information from underlying windowing system events into a structure that
can be consumed by the various components of each window manager. The last stage
is using  that windowing system  event information to perform  window management
actions, \textit{delegating work} to different parts of the program.

\subsection{Windowing System Events}

Both \textsc{wmRS} and \textsc{wmCPP} stipulate the existence of a \texttt{step}
method  in  their \texttt{Connection}  interfaces.  This  method is  responsible
for  converting  underlying windowing  system  information  into a  higher-level
event,   as  can   be  gleaned   from  their   signatures:  \mintinline{rust}{fn
step(&self) -> Option<Event>} (\textsc{wmRS}) and \mintinline{cpp}{Event step()}
(\textsc{wmCPP}), where in both cases  \texttt{Event} is some internally defined
structure of data.

\subsection{Internal Events}

% https://doc.rust-lang.org/book/ch06-00-enums.html

To structure event  data, \textsc{wmRS} uses Rust  \textit{enumerations}. A Rust
\mintinline{rust}{enum}  allows for  the  definition of  a  type by  enumerating
its  variants,  and, aside  from  encoding  \textit{meaning},  is also  able  to
encapsulate  \textit{data}\cite{therustbook}. A  small  part of  \textsc{wmRS}'s
\mintinline{rust}{Event} \mintinline{rust}{enum} looks as follows.

\begin{rustblock}
  pub enum Event {
    Mouse { event: MouseEvent },
    Key { keycode: KeyCode },
    MapRequest { window: Window, ignore: bool },
    FocusRequest { window: Window },
    CloseRequest { window: Window },
    ScreenChange,
    // ...
  }
\end{rustblock}

The   first  five   variants  are   what  are   known  as   \textit{struct-like}
variants   with  named   fields,   whereas   the  last   one   is  a   fieldless
variant  that   comprises  only  an   identifier\cite{therustreference}.  Behind
the   scenes,   each   \mintinline{rust}{enum}  instance   has   an   associated
integer   that  is   used  to   determine  the   underlying  concrete   variant,
known   as  a   \textit{discriminant}\cite{therustreference}.   As  such,   Rust
\mintinline{rust}{enum}s  are  \textit{tagged  unions},  where the  tag  is  the
discriminant.  This   also  means  that  the   size  of  each  instance   of  an
\mintinline{rust}{enum} is determined  by that \mintinline{rust}{enum}'s largest
variant. Rust's  \textit{default representation}  stores the discriminant  as an
\mintinline{rust}{isize}, although the compiler is free to make use of a smaller
type if the amount of variants permits it\cite{therustreference}.\\

% https://en.cppreference.com/w/cpp/utility/variant

A  similar  construct  in   \cpp  is  not  \textit{its}  \mintinline{cpp}{enum},
but   rather   \mintinline{cpp}{std::variant},   a  class   template   available
since  \texttt{C++17}  that  represents  a  type-safe  union\cite{cppstd}.  Just
as  with   Rust's  \mintinline{rust}{enum},  and  similar   to  regular  unions,
the  object   representation  of   the  concrete   alternative  held   inside  a
\mintinline{cpp}{std::variant}  instance  is   allocated  entirely  within  that
instance's object representation\cite{cppstd}. That  is, the alternative may not
allocate any additional dynamic memory\cite{cppstd}. As a result, as with Rust's
\mintinline{rust}{enum}, the  size of a  \mintinline{cpp}{std::variant} instance
is  dependent on  its  largest  alternative\cite{cppstd}. \textsc{wmCPP}  partly
represents its \mintinline{cpp}{Event} type as follows.

\begin{cppblock}
  typedef std::variant<
    std::monostate,
    Mouse,
    Key,
    MapRequest,
    FocusRequest,
    CloseRequest,
    ScreenChange,
    // ...
  > Event;
\end{cppblock}

Here,   each   concrete   alternative   is   a   \cpp   \mintinline{cpp}{struct}
containing   information    about   the    specific   event   that    is   being
represented  (e.g.  \mintinline{cpp}{typedef  struct  Mouse}  \mintinline{cpp}{{
MouseEvent  event;  }   Mouse;  }).  This  is  similar   to  what  \textsc{wmRS}
does   with    its   Rust    \mintinline{rust}{enum}s,   only    more   verbose,
non-inlined  (as  \mintinline{cpp}{std::variant}   cannot  deal  with  anonymous
\mintinline{cpp}{struct}s,  since  \cpp does  not  have  pattern matching),  and
therefore less structured and more tedious to work with.\\

% https://en.cppreference.com/w/cpp/utility/variant/monostate

An astute  reader may  have noticed that  the \texttt{Connection}  interfaces of
\textsc{wmRS}  and  \textsc{wmCPP} do  not  exactly  match. In  particular,  the
\texttt{step}  function's signatures  seem to  communicate different  things. In
\textsc{wmRS},  that  function  \textit{optionally}  yields  an  \texttt{Event},
wheres  in  \textsc{wmCPP},  it  simply  returns  an  \texttt{Event}.  In  \cpp,
when  using  a  \mintinline{cpp}{std::variant},   it  is  encouraged  to  encode
an   empty   alternative  using   \mintinline{cpp}{std::monostate}\cite{cppstd}.
The   same    behavior   \textit{could}    instead   have   been    encoded   as
\mintinline{cpp}{std::option<Event>}, where in this  case, the first alternative
of  \texttt{Event} is  omitted.  The  difference in  usage  mainly  lies in  the
\textit{visiting} of \texttt{Event}  alternatives, which we will  discuss in the
next section.

\subsection{Event Dispatch}

Once an  event has been  generated at the windowing  system and filtered  into a
structure that is to  be consumed by the window manager,  work must be delegated
to various handlers  that deal with specific types of  events. An \texttt{Event}
consequently encodes not  only the \textit{information} used  by these handlers,
but  also  which handler  is  responsible  for  that event.  \textsc{wmRS}  uses
Rust's principle pattern matching construct, \mintinline{rust}{match}, for event
dispatch. \textsc{wmRS}'s main event loop,  defined within a method belonging to
its \texttt{Model} structure, partly looks as follows.

\begin{rustblock}
  while self.running {
    if let Some(event) = self.conn.step() {
      match event {
        Event::Mouse { event, }
          => self.handle_mouse(event, /*...*/),
        Event::Key { keycode, }
          => self.handle_key(keycode, /*...*/),
        Event::MapRequest { window, ignore, }
          => self.handle_map_request(
            window, ignore
          ),
        Event::FocusRequest { window, }
          => self.handle_focus_request(window),
        Event::CloseRequest { window, }
          => self.handle_close_request(window),
        Event::ScreenChange
          => self.handle_screen_change(),
        // ...
      }
    }
  }
\end{rustblock}

The  \texttt{step} method  returns an  \mintinline{rust}{Option<Event>}, so  the
first thing to do is to check  whether a valid \texttt{Event} was retrieved from
the  windowing system  connection (i.e.,  whether the  event that  was generated
is  one  the  window  manager is  interested  in).  Rust's  \mintinline{rust}{if
let}  construct  is  syntactic  sugar  for a  match  statement  that  runs  code
when  the   value  matches   one  specific  pattern,   and  ignores   all  other
values\cite{therustbook}. Converting  the above \mintinline{rust}{if let}  to an
equivalent \mintinline{rust}{match} results in the following.

\begin{rustblock}
  match self.conn.step() {
    Some(event) => { /*...*/ },
    _ => (), // anything else => do nothing
  }
\end{rustblock}

Within the body  of the first match \textit{arm},  the \texttt{event} identifier
has been \textit{bound} to the value  that was passed in, \textit{if} that value
matches the pattern\cite{therustbook}. Here, that means that \textit{if} a valid
event is  returned from the  \texttt{step} method, that  concrete \texttt{Event}
instance  will be  bound to  the \texttt{event}  identifier within  the body  of
that  match arm.  The \texttt{\_}  placeholder is  used to  denote all  possible
\textit{remaining} values\cite{therustbook}. The Rust  compiler will assert that
all  possible  cases are  handled  within  a  match  statement, as  matches  are
exhaustive\cite{therustbook}.

Additionally  matching on  this \texttt{event},  all \textit{variants}  that can
possibly  exist for  its type  are enumerated,  binding identifiers  to each  of
the  fields in  the struct-like  variants  defined in  the \texttt{Event}  enum,
which are  then passed  on to  handler methods  within the  \texttt{Model} (e.g.
\mintinline{rust}{fn handle_focus_request(window: Window)}).
