%&latex

\section{Pragmatics}

\begin{csecframe}{Definition}

    \begin{enumerate}
        \itemsep1em
        \item \tbf{Syntax}\\
            Set of rules that define the \textit{structure} and \textit{composition}
            of allowable symbols into correct statements or expressions in the language

        \item \tbf{Semantics}\\
            The \textit{meaning} of these syntactically valid statements or expressions

        \item \tbf{Pragmatics}\\[3pt]
            \begin{aquote}{Robert D. Cameron, 2002}
                ...[T]he third general area of language description, referring
                to practical aspects of how constructs and features of a language may be used to
                achieve various objectives.
            \end{aquote}
    \end{enumerate}

\end{csecframe}

\begin{csecframe}{Example: Assignment Statements}

    \begin{enumerate}
        \item \tbf{Syntax} (\textit{structure})\\
            $x = y * 3;$

        \item \tbf{Semantics} (\textit{meaning})\\
            \begin{itemize}
                \itemsep.1em
                \item $x$\\
                Location in memory

                \item $y * 3$\\
                Computation of a value based on an expression

                \item $x = y * 3;$\\
                Store result of expression evaluation in location in memory
            \end{itemize}

        \item \tbf{Pragmatics} (\textit{purpose})\\[3pt]
            \textit{Which objectives} are assignment statements used \textit{for}?\\
            \begin{itemize}
                \itemsep.1em
                \item Setting up a temporary variable used to swap the values of two variables

                \item Modifying some part of a compound data structure

                \item ...
            \end{itemize}

    \end{enumerate}

\end{csecframe}
