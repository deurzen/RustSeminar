%&latex

\section{External Dependency Management}

\begin{frame}[c]{External Dependency Management}

    Practicalities of working with external code

    % two core aspects of working with an external body of code

    \begin{enumerate}
        \itemsep.3em

        \item \tbf{Package management}
            \begin{itemize}
                \item \textit{Availability} of external code
            \end{itemize}

        \item \tbf{Decoupling dependencies}
            \begin{itemize}
                \item \textit{Maintainability} of external code
            \end{itemize}

    \end{enumerate}

    \vfill

\end{frame}

\begin{secframe}{Package Management}

    \tbf{Managing the \underline{availability} of external code}

    \begin{itemize}

        \item The ability to \textit{aid} the programmer in assuring availability\\
            \begin{itemize}
                \item Automatically download and compile source code
                \item Built-in version control
                \item Conflict detection
            \end{itemize}

        \item Part of the \textit{ecosystem} of a language\\
            \begin{itemize}
                \item Installed with its compiler or development environment
            \end{itemize}

        \item A \textit{must} for any modern programming language\\

    \end{itemize}

    \vfill

\end{secframe}

\begin{frame}[c,fragile]{\underline{\cpp}: Package Management \hfill {\footnotesize \currentname}}

    \begin{itemize}

        \item \textit{No} official package manager\\

        \item \textit{Ad hoc} package management\\
            \begin{itemize}
                \item Third-party package management tools
                    \begin{itemize}
                        \item \textit{Conan}
                        \item \textit{Vcpkg}
                        \item \textit{build2}
                    \end{itemize}
                \item Custom configure and build scripts
                \item Let the user manage the dependencies themselves (e.g. through
                their distribution's package manager)
            \end{itemize}

        \item Example: \texttt{Make} script\\
\usemintedstyle{default}
\begin{minted}[fontsize=\scriptsize]{make}
  CXXFLAGS  := -std=c++20 -march=native -O3
  LDFLAGS   := `pkg-config --libs x11 xrandr` -flto
\end{minted}
\begin{minted}[fontsize=\scriptsize]{make}
  SRC_FILES := $(wildcard src/*.cc)
  OBJ_FILES := $(patsubst src/%.cc,obj/%.o,${SRC_FILES})
\end{minted}
\begin{minted}[fontsize=\scriptsize]{make}
  all: ${OBJ_FILES}
      g++ ${OBJ_FILES} ${LDFLAGS} -o bin/wmCPP
\end{minted}
\begin{minted}[fontsize=\scriptsize]{make}
  obj/%.o: src/%.cc
      g++ ${CXXFLAGS} -MMD -c $< -o $@
\end{minted}
\usemintedstyle{igor}

    \end{itemize}

\end{frame}

\begin{frame}[c,fragile]{\underline{Rust}: Package Management \hfill {\footnotesize \currentname}}

    \begin{itemize}

        \item \textit{Cargo}, Rust's official package manager\\
            \begin{itemize}
                \item Automatically downloads and compiles dependencies
                \item A Rust project is a Cargo \textcolor{red}{\textit{package}}
                \item A \textcolor{red}{package} is a collection of \textit{source files} plus a \textcolor{orange}{\textit{manifest}} file
                \item The \textcolor{orange}{manifest} file describes the package's \textit{meta-information}, \textit{dependencies}, and a set of \textit{target \textcolor{ForestGreen}{crates}}
                \item A \textcolor{ForestGreen}{crate} represents a \textit{library} or \textit{binary executable} program
            \end{itemize}

        \item Example: \texttt{Cargo.toml} manifest file\\
\usemintedstyle{default}
\begin{minipage}{.55\linewidth}
\begin{minted}[fontsize=\scriptsize]{rust}
   [package]
   name = "wmRS"
   version = "0.1.0"
   edition = "2018"
   license = "BSD3"
   default-run = "core"
   description = """
   An ICCCM & EWMH compliant X11
   reparenting, tiling window manager,
   written in Rust
   """
\end{minted}
\end{minipage}%
\begin{minipage}{.35\linewidth}
\begin{minted}[escapeinside=||,fontsize=\scriptsize]{rust}
  |\bfseries{\textcolor{ForestGreen}{[lib]}}|
  name = "winsys"
  path = "src/winsys/mod.rs"
  |\bfseries{\textcolor{ForestGreen}{[[bin]]}}|
  name = "core"
  path = "src/core/main.rs"
  |\bfseries{\textcolor{ForestGreen}{[[bin]]}}|
  name = "client"
  path = "src/client/main.rs"
  [dependencies]
  x11rb = "0.8.0"
\end{minted}
\end{minipage}
\usemintedstyle{igor}

    \end{itemize}

    \vfill

\end{frame}

\begin{secframe}{Decoupling Dependencies}

    \tbf{Managing the \underline{maintainability} of external code}

    \begin{itemize}

        \item The ability to \textit{decouple} own code from external code\\
            \begin{itemize}
                \item Changes to own code don't affect interface with external code
                \item Changes to external code \textit{only} affect inerface with external code
            \end{itemize}

        \item When external code changes:\\
            \begin{itemize}
                \item Only interface with external code needs to be recompiled
            \end{itemize}

        \item When own code changes:\\
            \begin{itemize}
                \item Only own code needs to be recompiled
            \end{itemize}

    \end{itemize}

    \vfill

\end{secframe}

\begin{secframe}{Decoupling Dependencies}

    \tbf{Decouple \underline{window manager} from \underline{windowing system}}

    \begin{enumerate}

        \item Hide the connection with the windowing system behind an \textit{interface}\\
            \begin{itemize}
                \item Provide \textit{abstraction} and \textit{encapsulation}
                \item Describe \textit{common behavior}
                \item \textit{Usage} is \textit{agnostic} of concrete implementation
            \end{itemize}

        \item Implement the interface for \textit{each} targeted windowing system\\
            \begin{itemize}
                \item Implement the interface to target the \textit{X Window System}
                \item Implement the interface to target \textit{Wayland}
                \item Implement the interface to target the \textit{Desktop Window Manager} (Windows)
                \item Implement the interface to target the \textit{Quartz Compositor} (macOS)
            \end{itemize}

        \item Have the window management logic call into the interface\\

    \end{enumerate}

    \vfill

\end{secframe}

\begin{frame}[fragile]{\underline{Rust}: Decoupling with Traits \hfill {\footnotesize \currentname}}

    \tbf{1. Hide the connection with the windowing system behind a \underline{trait}}

    \begin{itemize}

        \item Comparable to, \textit{but not the same as}, the concept of an OOP \textit{interface}\\
            \begin{itemize}
                \item Implementation does not require changes to the implementor
                \begin{itemize}
                    \item Traits can be implemented on \textit{external} code
                    \item No ambiguity when two implemented traits share method name and prototype
                \end{itemize}
            \end{itemize}

        \item Can define \textit{stateless} default implementations\\

        \item \textit{Zero-overhead} collection of methods\\[3pt]
            { \footnotesize
            \begin{aquote}{\footnotesize Bjarne Stroustrup}
                What you don't use, you don't pay for [Stroustrup, 1994]. And further: What you do use, you couldn't hand code any better.
            \end{aquote} }

    \end{itemize}

    \vfill

\end{frame}

\begin{frame}[fragile]{\underline{Rust}: Decoupling with Traits \hfill {\footnotesize \currentname}}

    \tbf{1. Hide the connection with the windowing system behind a \underline{trait}}

    \begin{itemize}

        \item No inheritance, only implementation\\
            \begin{itemize}
                \item No downcasting or reference casting
            \end{itemize}

        \item Declared for some (at declare-time) unknown type \texttt{Self}\\
            \begin{itemize}
                \item When implemented \texttt{Self} becomes the implementing type
            \end{itemize}

        \item Example: \textsc{wmRS}'s \rsin{Connection} trait:\\
\begin{minted}[fontsize=\scriptsize]{rust}
   pub trait Connection {
     fn step(&self) -> Option<Event>;
     fn move_window(&self, window: Window, pos: Pos);
     fn resize_window(&self, window: Window, dim: Dim);
     fn close_window(&self, window: Window);
     // ...
   }
\end{minted}

    \end{itemize}

    \vfill

\end{frame}

\begin{frame}[fragile]{\underline{Rust}: Decoupling with Traits \hfill {\footnotesize \currentname}}

    \tbf{2. Implement the trait for \textit{each} targeted windowing system}

    \begin{itemize}

        \item Example: \textsc{wmRS}'s \rsin{XConnection} structure:\\
\begin{minted}[fontsize=\scriptsize]{rust}
  use x11rb::connection;
\end{minted}
\begin{minted}[fontsize=\scriptsize]{rust}
  pub struct XConnection<'conn, Conn: connection::Connection> {
    conn: &'conn Conn,
    // ...
  }
\end{minted}
\begin{minted}[fontsize=\scriptsize]{rust}
  impl<'conn, Conn: connection::Connection> Connection
    for XConnection<'conn, Conn>
  {
    fn step(&self) -> Option<Event> { /* ... */ }
    // ...
  }
\end{minted}

        \item \rsin{x11rb}: Rust library to interact with the X Window System\\
            \begin{itemize}
                \item External dependency
                \item Contains Rust bindings to interact with the X server
            \end{itemize}

    \end{itemize}

    \vfill

\end{frame}

\begin{frame}[fragile]{\underline{Rust}: Decoupling with Traits \hfill {\footnotesize \currentname}}

    \tbf{3. Have the window management logic call into the interface}

    \begin{itemize}

        \item Example: \textsc{wmRS}'s core window manager logic:\\
\begin{minted}[fontsize=\scriptsize]{rust}
  pub struct Model<'model> {
    conn: &'model mut dyn Connection,
    // ...
  }
\end{minted}

        \item \rsin{Model} \textit{contains} a reference to \textit{some} \rsin{Connection} implementor\\

        \item The trait methods of this implementor are called where needed\\
            \begin{itemize}
                \item Dynamic dispatch
                    \begin{itemize}
                        \item Concrete method to call is looked up at \textit{runtime}
                    \end{itemize}
                \item Static dispatch
                    \begin{itemize}
                        \item Concrete method to call is baked into the binary
                    \end{itemize}
            \end{itemize}

    \end{itemize}

    \vfill

\end{frame}
