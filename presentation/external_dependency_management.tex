%&latex

\section{External Dependency Management}

\begin{frame}[c]{External Dependency Management}

    Practicalities of working with external code

    % two core aspects of working with an external body of code

    \begin{enumerate}
        \itemsep.3em

        \item \tbf{Package management}
            \begin{itemize}
                \item \textit{Availability} of external code
            \end{itemize}

        \item \tbf{Decoupling dependencies}
            \begin{itemize}
                \item \textit{Maintainability} of external code
            \end{itemize}

    \end{enumerate}

    \vfill

\end{frame}

\begin{secframe}{Package Management}

    \begin{itemize}

        \item The ability to \textit{aid} the programmer in managing external code\\
            \begin{itemize}
                \item Automatically downloading a dependency's source code
                \item Built-in version control
                \item Conflict detection
            \end{itemize}

        \item Part of the \textit{ecosystem} of a language\\
            \begin{itemize}
                \item Installed with its compiler or development environment
            \end{itemize}

        \item A \textit{must} for any modern programming language\\

    \end{itemize}

    \vfill

\end{secframe}

\begin{frame}[fragile]{\underline{\cpp}: Package Management \hfill {\footnotesize \currentname}}

    \begin{itemize}

        \item \textit{No} official package manager\\

        \item \textit{Ad hoc} package management\\
            \begin{itemize}
                \item Third-party package management tools
                \item Custom configure and build scripts
                \item Let the user manage the dependencies themselves (e.g. through
                their distribution's package manager)
            \end{itemize}

        \item Example: \texttt{Make} script\\
\begin{minted}[fontsize=\scriptsize]{make}
   CXXFLAGS = -std=c++17 -march=native -O3
   LDFLAGS = `pkg-config --libs x11 xrandr` -flto

   obj/%.o: src/%.cc
     g++ ${CXXFLAGS} -MMD -c $< -o $@
   all: obj/%.o
     g++ $< ${LDFLAGS} -o bin/wmCPP
\end{minted}

    \end{itemize}

\end{frame}

\begin{frame}[c,fragile]{\underline{Rust}: Package Management \hfill {\footnotesize \currentname}}

    \begin{itemize}

        \item \textit{Cargo}, Rust's official package manager\\
            \begin{itemize}
                \item Automatically downloads and compiles dependencies
                \item A Rust project is a Cargo \textit{package}
                \item A package is a collection of \textit{source files} plus a \textit{manifest file}
                \item The manifest file describes the package's \textit{meta-information}, \textit{dependencies}, and a set of \textit{target crates}
                \item A crate represents a \textit{library} or \textit{binary executable} program
            \end{itemize}

        \item Example: \texttt{Cargo.toml} manifest file\\
\begin{minted}[fontsize=\scriptsize]{rust}
   [package]
   name = "wmRS"
   version = "0.1.0"
   edition = "2018"
   license = "BSD3"
   default-run = "wmRS"
   description = """
   An ICCCM & EWMH compliant X11 reparenting,
   tiling window manager, written in Rust
   """
\end{minted}

    \end{itemize}

\end{frame}

\begin{csecframe}{Decoupling Dependencies}

    \begin{itemize}

        \item What is a \textit{Windowing System}?\\
            \begin{itemize}
                \item System software
                \item Responsible for providing graphical primitives to construct and present GUIs
                \item Render applications' windows' contents
                \item Main dependency of the window manager
            \end{itemize}

        \item The window manager's implementation is made \textit{agnostic} to that of the windowing system\\
            \begin{itemize}
                \item We create a \textit{library}, \texttt{winsys}, that represents an abstraction above and wrapper around the API into the windowing system
                \begin{itemize}
                    \item The library defines an \textit{interface} that outlines desired behavior
                    \item The interface is \textit{implemented} for each supported windowing system
                \end{itemize}
                \item The connection with windowing system is \textit{decoupled} from the implementation of the window manager
            \end{itemize}

    \end{itemize}

    \vfill

\end{csecframe}

\begin{frame}[c,fragile]{\underline{Rust}: Decoupling using Traits \hfill {\footnotesize \currentname}}
    To define an interface that represents the connection with the windowing system, we use a Rust \textit{trait}
    \begin{itemize}

        \item Traits are \textit{zero-overhead} collections of methods that are:\\
            \begin{itemize}
                \item Declared for \textit{some} type \texttt{Self}
                \item Most often used to implement \textit{shared behavior}
            \end{itemize}

        \item Traits most closely resemble the concept of \textit{interfaces} from other languages\\

        \item Traits can provide a \textit{default implementation} for their defined methods\\

        \item Example: \texttt{wmRS}'s \texttt{Connection} trait\\
\begin{minted}[fontsize=\scriptsize]{rust}
   pub trait Connection {
     fn step(&self) -> Option<Event>;
     fn move_window(&self, window: Window, pos: Pos);
     fn resize_window(&self, window: Window, dim: Dim);
     fn close_window(&self, window: Window);
     // ...
   }
\end{minted}

    \end{itemize}

\end{frame}

\begin{frame}[c,fragile]{\underline{Rust}: Decoupling using Traits \hfill {\footnotesize \currentname}}
    We implement the \texttt{Connection} trait to target a \textit{specific} windowing system
    \begin{itemize}

        \item Example: \textsc{wmRS}'s \texttt{Connection} implementation for the X Window System\\
\begin{minted}[fontsize=\scriptsize]{rust}
   use x11rb::connection;

   pub struct XConnection
     <'conn, Conn: connection::Connection>
   {
     conn: &'conn Conn,
     // ...
   }
   impl<'conn, Conn: connection::Connection>
     Connection for XConnection<'conn, Conn>
   {
     fn step(&self) -> Option<Event> { /* ... */ }
     // ...
   }
\end{minted}

    \end{itemize}

\end{frame}
