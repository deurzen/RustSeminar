%&latex
\begin{abstract}
% https://www2.cs.sfu.ca/~cameron/Teaching/383/syn-sem-prag-meta.html

In comparing  and discussing programming  languages (and natural  languages, for
that matter), not only \textit{syntax} and \textit{semantics} are of importance.
\textit{Pragmatics} is  the third  general area  of language  description, which
deals with the  practical aspects of how language constructs  and features allow
its users  to achieve various objectives[CITE].  In this paper, we  will look at
the pragmatics of both Rust and C\texttt{++}. Specifically, we will be comparing
the  languages in  their  ability  to be  used  as a  tool  to  write medium  to
large  \textit{system programs}.  As a  case study,  we will  be discussing  two
implementations of  a complete ICCCM  and EWMH compliant  top-level reparenting,
tiling window manager, built on top of the X Window System: one written in Rust,
and the other in C\texttt{++}.

\end{abstract}
