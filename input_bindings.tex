%&latex
\section{Input Bindings}
% TODO: - complex feature of which the implementation constructs trickle into
%         many parts of the program (large fan-out)
%       - data structures and language features used to represent key bindings
%       - verbose (kranewm, C++) vs use of macros (wzrd, Rust)

% https://locka99.gitbooks.io/a-guide-to-porting-c-to-rust/content/features_of_rust/macros.html

Key and mouse bindings are the central point of interaction between the user and
the window manager. At a high level, as the name would suggest, they work by
\textit{binding} to sets of \textit{input states}. These input states typically
comprise keyboard (i.e., \textit{which keys are pressed down?}) and mouse
(\textit{which mouse buttons are pressed down?}) inputs, but can in principle
come from any kind of peripheral input source, such as sensors. Since input
readings are hardware and, transitively, platform dependent, both \wmrs and
\wmcpp rely on the connection with the windowing system to facilitate these
bindings. That is---as can be seen from the \texttt{Connection} interfaces given
in the examples above---both window managers receive specific events (e.g.
\texttt{Key} and \texttt{Mouse}) that communicate that a specific binding has
been activated.

\rssubsection{Storing and Retrieving Callable Objects}

To attach window management activities to these key and mouse bindings,
\textsc{wmRS} uses a Rust \rsin{HashMap} to map resp. mouse and key input state
abstractions to \textit{closures} that call into part of the \texttt{Model}.
Mouse input abstractions look something like the following (parts redacted for
clarity):

\begin{rustblock}
  #[derive(Clone, Copy, PartialEq, Eq, Hash)]
  pub enum MouseEventKind {
    Press, Release, Motion,
  }
\end{rustblock}

\begin{rustblock}
  #[derive(Clone, Copy, PartialEq, Eq, Hash)]
  pub enum Button {
    Left, Middle, Right, ScrollUp, // ...
  }
\end{rustblock}
\begin{rustblock}
  #[repr(u8)]
  #[derive(Clone, Copy, PartialEq, Eq, Hash)]
  pub enum Modifier {
    Ctrl = 1 << 0, Shift = 1 << 1,
    Super = 1 << 2, Alt = 1 << 3,
    // ...
  }
\end{rustblock}

\begin{rustblock}
  #[derive(PartialEq, Eq)]
  pub struct MouseInput {
    pub button: Button,
    pub modifiers: HashSet<Modifier>,
  }
\end{rustblock}

\begin{rustblock}
  #[derive(PartialEq, Eq, Hash)]
  pub struct MouseEvent {
    pub kind: MouseEventKind,
    pub input: MouseInput,
    pub window: Option<Window>,
  }
\end{rustblock}

Since we are dealing with hardware and platform dependencies, \textsc{wmRS}
again relies on the connection with the windowing system to establish the
mapping from concrete underlying window system input encodings to the
abstractions above. In particular, we expect a \texttt{MouseEvent} to be
fired (and propagated up through \texttt{Connection::step}) whenever a
button is pressed. The event's input \textit{modifiers} are contained within
a Rust \texttt{HashSet}, as ordering is irrelevant and possible duplicates
are redundant. Its \texttt{window} field is to optionally contain the unique
identifier of the window that was clicked on (if any). The representation of
key events is similar to the above, but instead of mouse buttons it includes
non-modifier key identifiers.

The actual mapping between these input abstractions and hooks into the window
manager's \texttt{Model} then looks as follows:

\begin{rustblock}
  pub type KeyAction = Box<
    dyn FnMut(&mut Model<'_>)
  >;
\end{rustblock}
\begin{rustblock}
  pub type MouseAction = Box<
    dyn FnMut(&mut Model<'_>, Option<Window>)
  >;
\end{rustblock}
\begin{rustblock}
  pub type KeyBindings = HashMap<
    KeyInput, KeyAction
  >;
\end{rustblock}
\begin{rustblock}
  pub type MouseBindings = HashMap<
    MouseInput, MouseAction
  >;
\end{rustblock}

These are simple \textit{type alias} declarations, giving existing (in
this case compound) types a new name. The first two represent closures
that hook into (i.e. take as parameter) a \textit{mutable}---or, more
semantically explicative: \textit{exclusive}---reference to the model
instance\cite{therustbook}. \rsin{FnMut} is a trait that describes the calling
of a function that is allowed to change (\textit{mutate}) state; it permits
its implementors to use call syntax, such that they can act and look like
regular function pointers. \rsin{Box} \textit{boxes} up a value, storing it
on the heap and yielding a pointer to it (under the hood). Specifically,
here, that means that \textit{some} type that implements the \rsin{FnMut}
trait and takes a mutable reference to a model instance (\rsin{dyn FnMut(&mut
Model<'_>)}) is boxed up and stored on the heap for future reference. We
require the use of \rsin{Box}, because we want to be able to store \textit{any
implementor} of the \rsin{FnMut} trait. To do so, Rust will again rely on
dynamic dispatch\cite{rustclosureshard}. The advantage of this approach is
that it facilitates the changing of input bindings at runtime, allowing for
multi-mode hierarchical input bindings, as well as dynamic user-initiated
binding customizations.

To be able to use \rsin{KeyInput} and \rsin{MouseInput} as keys to the
\rsin{KeyBindings} and \rsin{MouseBindings} \rsin{HashMap}s, they must satisfy
several conditions. These conditions are expressed through the trait bounds
\rsin{PartialEq}, \rsin{Eq}, and \rsin{Hash}, respectively defining partial
equivalence and equivalence relations, enabling the comparison for (partial)
equality, and a hash function, allowing instances of its implementors to be
\textit{hashed} with an implementation of \rsin{Hasher}\cite{therustbook,
therustreference}. \rsin{PartialEq} and \rsin{Eq} can be automatically
implemented for \rsin{KeyInput} and \rsin{MouseInput} by making use of Rust's
\rsin{derive} attribute. This attribute asks the compiler to provide basic
implementations for built-in (and even self-written, if an implementation is
first defined using \textit{procedural macros}) traits, given that the trait and
the type requiring the implementation are \textit{derivable}\cite{therustbook,
therustreference}. For \rsin{PartialEq} and \rsin{Eq} in particular, all
fields of a to-be-derived compound type \textit{must} already implement
\rsin{PartialEq} and \rsin{Eq}, which is the case for \rsin{MouseEvent}'s
fields given in the example above. The same condition holds for \rsin{Hash},
although \rsin{HashSet}, the type of \rsin{MouseInput}'s \rsin{modifiers}
field, is not automatically derivable\cite{therustbook, therustreference}.
To satisfy the \rsin{Hash} trait bound for \rsin{MouseInput}, it must
therefore be defined manually. To this end, each variant of \textsc{wmRS}'s
\rsin{Modifier} enumeration is represented by an unsigned 8-bit integer value
(\rsin{#[repr(u8)]}). The values are assigned in such a manner that the
\rsin{enum} acts as a collection of \textit{bit flags}, resolving each variant
\textit{and} the logical disjunction of any combination of variants into a
unique value. \rsin{MouseInput}'s \rsin{Hash} implementation then looks as
follows:

\begin{rustblock}
  impl Hash for MouseInput {
    fn hash<H: Hasher>(&self, state: &mut H) {
      self.button.hash(state);
      self.modifiers.iter()
        .fold(0u8, |acc, &m| acc | m as u8)
        .hash(state);
    }
  }
\end{rustblock}

Hashing works relative to an object that implements the \rsin{Hasher} trait
(\rsin{<H: Hasher>})\cite{therustbook}. The methods required by \rsin{Hasher}
are to contain the logic of the hashing function in use, and an instance of
an implementor of the trait will usually represent some \textit{state} that
changes as data is being hashed\cite{therustbook, rusttrickhashmap}. The final
state of such an instance then represents the hashed value\cite{therustbook,
rusttrickhashmap}. Hashing structures that expose this behavior are called
\textit{streaming} hashers\cite{buildingahashmap}. To obtain the hashed value
of a \rsin{MouseInput} instance, we change, in ordered succession, the state
of the \rsin{Hasher} object by supplying the hasher the \rsin{button} field's
value, followed by the logical disjunction of all of the \rsin{Modifier}
variants contained in the \rsin{modifiers} field. When looking for the
matching value of a supplied key in the \rsin{HashMap}, Rust will use the key
instance's \rsin{hash}ed value to compute the \textit{''bucket``} from which
to start the search, and \rsin{eq} (the comparison method implemented through
\rsin{PartialEq} and \rsin{Eq}) to actually perform the search.

To finally add a new mouse binding, we insert an input/action pair into the
\rsin{MouseBindings} \rsin{HashMap}, as follows:

\begin{rustblock}
  let mut mouse_bindings = HashMap::new();

  mouse_bindings.insert(
    MouseInput {
      button: Button::Middle,
      modifiers: {
        let mut modifiers = HashSet::with_capacity(3);
        modifiers.insert(Modifier::Ctrl);
        modifiers.insert(Modifier::Shift);
        modifiers.insert(Modifier::Super);
        modifiers
      }
    },
    Box::new(|model: &mut Model, w: Option<Window>| {
      if let Some(window) = w {
        model.set_fullscreen_window(window);
      }
    })
  );
\end{rustblock}

To retrieve a mouse binding and call its function, a handler method within
\textsc{wmRS}'s \rsin{Model} gets passed in a \rsin{MouseEvent} that contains
information about the button and modifiers that had been pressed. The contained
logic looks somewhat as follows:

\begin{rustblock}
  fn handle_mouse(
      &mut self,
      event: MouseEvent,
      mouse_bindings: &MouseBindings,
  ) {
    // ...
    if let Some(action)
        = mouse_bindings.get(&event.input)
    {
      action(self, event.window);
    }
    // ...
  }
\end{rustblock}

Population of and retrieval from \textsc{wmRS}'s \rsin{KeyBindings}
\rsin{HashMap} is done in similar fashion.

\cppsubsection{std::function and std::unordered\_map}

To achieve an analogous mouse and key binding setup in \textsc{wmCPP}, we
first create \cpp{}-equivalent scoped enumerations (\cppin{enum class}) for
\cppin{MouseEventKind} and \cppin{Button}, and an unscoped enumeration (C-style
\cppin{enum}) for \cppin{Modifier}. Unfortunately, an \cppin{enum class} cannot
be used to represent modifier keys, as that construct is not effective in
encoding bit flags. The reason for this is that the logical disjunction of
two scoped bit flag enumerators would not resolve into another valid scoped
enumerator. Additionally, the following \cppin{struct}s are constructed to
represent \cppin{MouseInput} and \cppin{MouseEvent}:

\begin{cppblock}
  struct MouseInput final {
    const Button button;
    const std::unordered_set<Modifier> modifiers;
  };

  struct MouseEvent final {
    MouseEventKind kind;
    MouseInput input;
    std::optional<Window> window;
  };
\end{cppblock}

Just as in \textsc{wmRS}, a hash map construct is used to map input
abstractions to closures that call into the \cppin{Model} object. Specifically,
\cppin{std::unordered_map}---and not \cppin{std::map}, which \textit{orders} its
entries by key and is therefore (in this case needlessly) less efficient---will
take as key arbitrary objects of type \cppin{T} that provide overloads for
\cppin{operator==} and \cppin{std::hash<T>::operator()}. The first is the
equality comparison operator, comparable to Rust's \rsin{PartialEq::eq} trait
method. The \cppin{std::hash<T>} template's \cppin{operator()} overload
implements a hash function that, when called, converts instances of type
\cppin{T} into an unsigned integer (\cppin{std::size_t}) that represents the
hashed value. \textsc{wmCPP}'s implementation for both overloads looks as
follows:

\begin{cppblock}
inline bool
operator==(const MouseInput& lhs, const MouseInput& rhs)
{
  return lhs.button == rhs.button
    && lhs.modifiers == rhs.modifiers;
}

namespace std
{
  template <>
  struct hash<MouseInput>
  {
    std::size_t
    operator()(const MouseInput& input) const
    {
      std::size_t button_hash
        = std::hash<Button>()(input.button);

      std::size_t modifiers_hash
        = std::hash<std::size_t>()(
          std::accumulate(input.modifiers.begin(),
            input.modifiers.end(),
            static_cast<Modifier>(0),
            std::bit_or<Modifier>()
        )
      );

      return button_hash ^ modifiers_hash;
    }
  };
}
\end{cppblock}

Conceptually, the same is happening here as in \textsc{wmRS}'s overloads for
\rsin{PartialEq::eq} and \rsin{Hash::hash}. Comparison of \cppin{MouseInput}
objects is simply done based on their constituent fields. To obtain the hashed
value of such an object, we take the \textit{exclusive or} of the hashed values
of both the \cppin{button} field and the logical disjunction of the values in
the \cppin{modifiers} \cppin{std::unordered_set}. The \cppin{std::accumulate}
function is similar to Rust's \rsin{Iterator::fold} method that \textit{folds}
every element of a collection into an accumulator by applying some operation.
In both cases, this operation is the logical disjunction (bitwise or), and the
accumulator is initialized to zero. The reason \cppin{Modifier} needed to be an
unscoped enumeration is due to this particular use of the \cppin{std::bit_or}
operation. However, something is still missing. Since, out of the box, the
bitwise or operator is not defined on \cppin{Modifier}'s enumerators, we must
first define it ourselves:

\begin{cppblock}
  inline Modifier
  operator|(Modifier lhs, Modifier rhs)
  {
    return static_cast<Modifier>(
      static_cast<std::size_t>(lhs)
        | static_cast<std::size_t>(rhs)
    );
  }
\end{cppblock}

% convey

\subsection{Input Bindings: Remarks}

Whereas Rust uses streaming hashers, consuming an arbitrary amount of
data byte-per-byte until it is told to yield a hashed summary, \cpp{}'s
\cppin{std::hash<T>::operator()} relies on so-called \textit{hash
coding}\cite{buildingahashmap}. Hash coding is a lot less convenient---as
can clearly be seen in the examples provided above---as complex data types
are expected to be reduced into a single integer within the body of a single
function\cite{buildingahashmap}. Moreover, \cpp does not allow for a central
(e.g. on a per \rsin{HashMap} instance basis) hashing algorithm to be defined,
which Rust does allow for through its \rsin{Hasher} trait\cite{therustbook,
therustreference}.

In \textsc{wmRS}, simply registering a single mouse binding is quite verbose.
To aid in populating the \rsin{HashMap}, and to increase readability and
programmer convenience, Rust's \textit{macros} can be used. Specifically,
\textit{declarative macros} facilitate a form of \textit{metaprogramming}---a
way of \textit{writing code with code}\cite{therustbook, therustreference}.

% Discuss {Build,}Hashers -> more efficient as we have known-to-be-distinct keys (input bindings),
% Pass-through hasher
% therefore we don't require complex collision safety etc.
