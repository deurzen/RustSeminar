\documentclass[conference]{IEEEtran}
% \IEEEoverridecommandlockouts
% The preceding line is only needed to identify funding in the first footnote. If that is unneeded, please comment it out.
\usepackage{cite}
\usepackage{amsmath,amssymb,amsfonts}
\usepackage{algorithmic}
\usepackage{graphicx}
\usepackage{textcomp}
\usepackage{xcolor}
\usepackage{blindtext}
\def\BibTeX{{\rm B\kern-.05em{\sc i\kern-.025em b}\kern-.08em
    T\kern-.1667em\lower.7ex\hbox{E}\kern-.125emX}}
\begin{document}

\title{Pragmatics of Rust and C\texttt{++}:\\The implementation of a window manager\\
}

\author{\IEEEauthorblockN{Max van Deurzen}
\IEEEauthorblockA{
\textit{Technische Universität München}\\
Munich, Germany \\[0.08cm]
\texttt{m.deurzen@tum.de}}
}

\maketitle

\begin{abstract}
% https://www2.cs.sfu.ca/~cameron/Teaching/383/syn-sem-prag-meta.html

In comparing  and discussing programming  languages (and natural  languages, for
that matter), not only \textit{syntax} and \textit{semantics} are of importance.
\textit{Pragmatics} is  the third  general area  of language  description, which
deals with the  practical aspects of how language constructs  and features allow
its users  to achieve  various objectives. In  this paper, we  will look  at the
pragmatics of  both Rust  and C\texttt{++}. Specifically,  we will  be comparing
the  languages in  their  ability  to be  used  as a  tool  to  write medium  to
large  \textit{system programs}.  As a  case study,  we will  be discussing  two
implementations of  a complete ICCCM  and EWMH compliant  top-level reparenting,
tiling window manager, built on top of the X Window System: one in Rust, and the
other in C\texttt{++}.
\end{abstract}

\begin{IEEEkeywords}
	Pragmatics, Rust, C\texttt{++}, Window Manager
\end{IEEEkeywords}

\section{Introduction}
{ \textcolor{gray}\blindtext }

\section{External Dependency Management}
% TODO: - Xlib (C++ impl.) and XCB (Rust impl.) dependencies
%       - wrapper {class,struct} (resp.) and decoupling from external
%         dependencies
%       - {lack of,} (resp.) ease of using external libraries
{ \textcolor{gray}\blindtext }

\subsection{Portability}
% TODO: - package manager(s), manully by developer (C++) vs Cargo (Rust)
%       - discrepancies between operating systems, platforms
{ \textcolor{gray}\blindtext }

\subsection{Versioning}
% TODO: - manual (C++) vs automatic (Rust) versioning
{ \textcolor{gray}\blindtext }

\section{Main Event Loop}
% TODO: - event dynamic kranewm (C++ impl.) vs wzrd (Rust impl.)
%       - event loop constructs and feature usage kramewm vs wzrd
{ \textcolor{gray}\blindtext }

\subsection{Internal Events}
% TODO: - data structures and language features used to represent events
{ \textcolor{gray}\blindtext }

\subsection{Event Dispatch}
% TODO: - language constructs used to handle event dispatch
{ \textcolor{gray}\blindtext }

\section{Communication with the Environment}
% TODO: - complex feature of which the implementation constructs trickle into
%         many parts of the program (large fan-out)
%       - data structures and language features used to represent key bindings
%       - verbose (kranewm, C++) vs use of macros (wzrd, Rust)
{ \textcolor{gray}\blindtext }

\subsection{Inter-Process Communication}
{ \textcolor{gray}\blindtext }

\section{Key Bindings}
% TODO: - complex feature of which the implementation constructs trickle into
%         many parts of the program (large fan-out)
%       - data structures and language features used to represent key bindings
%       - verbose (kranewm, C++) vs use of macros (wzrd, Rust)
{ \textcolor{gray}\blindtext }

\section{Clients}
% TODO: - main data structure that represents a window under window manager
%         control
%       - wrapper around X11 window
%       - directly manipulated by practically all parts of the program
%       - reason for pointer (C++) vs id (Rust) representation
{ \textcolor{gray}\blindtext }

\subsection{Reference Management}
% TODO: - difference between using direct/smart pointer (C++) vs non-pointer
%         indirection (Rust)
{ \textcolor{gray}\blindtext }

\subsection{State}
{ \textcolor{gray}\blindtext }

\section{Workspaces}
% TODO: - data structures and language features used
{ \textcolor{gray}\blindtext }

\section{Conclusion}
{ \textcolor{gray}\blindtext }

\begin{thebibliography}{00}
\bibitem{b1} { \textcolor{gray}\blindtext }
\end{thebibliography}

\end{document}
