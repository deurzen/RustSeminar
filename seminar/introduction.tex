%&latex
\section{Introduction}\label{introduction}

Both Rust and \cpp are \textit{system programming} languages, as they offer
high performance and ease of access to the underlying hardware. On top of that,
Rust also offers novel memory safety guarantees that cater well to low-level
system programming needs, as memory safety errors have historically been
the main cause of security exploits. In this seminar, we will be discussing
the \textit{pragmatics} of Rust and \cpp. A language's pragmatics pertains
to the \textit{purpose} of its language constructs, and the usability of
those constructs to achieve certain objectives. In general, the pragmatics
of a language construct can be boundless. Take, for instance, the assignment
statement. Its common purposes could be modifying part of a compound data
structure, or using a temporary variable to swap two values. To properly
compare the pragmatics of Rust and \cpp, instead of scrutinizing the purpose
of \textit{all} language constructs in turn, we will be evaluating both
languages in their ability to specifically achieve a \textit{single, common
objective}: the implementation of a window manager. In particular, we identify
four core phases in the design of such software. In section \ref{extdepman}, we
will discuss each language's capacity in dealing with external dependencies.
Particularly, we will look at which language features aid the programmer in
managing external bodies of code, and how these dependencies can be abstracted
upon and decoupled from. In section \ref{maineventloop}, we will compare the
languages' capabilities in wrapping external events, and in delegating work
based on the specific kind of events generated. Section \ref{inputbindings} will
cover language constructs that facilitate the storage and retrieval of callable
objects, such as to allow for input bindings and event dispatch. Finally, in
section \ref{clients}, each language's ability to allow for the mutation of
distributed state is discussed.

% achieve the same with the same approach
% -> same that affords ... a different approach
