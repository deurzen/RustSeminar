%&latex
\begin{abstract}

In comparing and discussing programming languages (and natural languages, for
that matter), not only \textit{syntax} and \textit{semantics} are of importance.
\textit{Pragmatics} is the third general area of language description, which
deals with the practical aspects of how language constructs and features allow
its users to achieve various objectives\cite{pragmatics}. In this paper, we
will look at the pragmatics of both Rust and \cpp. Specifically, we will be
comparing the languages in their ability to be used as a tool to write medium
to large \textit{system programs}. As a case study, we will be discussing
two implementations of an ICCCM\cite{icccm} and EWMH\cite{ewmh} compliant
top-level reparenting, tiling window manager, built on top of the X Window
System\cite{x11}: one written in Rust\cite{wzrd}, and the other in \cpp.

\end{abstract}
