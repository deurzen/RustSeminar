%&latex
\section{Clients}
% TODO: - main data structure that represents a window under window manager
%         control
%       - wrapper around X11 window
%       - directly manipulated by practically all parts of the program
%       - reason for pointer (C++) vs id (Rust) representation

So far, the notion of a \textit{window} was used throughout without defining
what exactly it is to entail. In both \textsc{wmRS} and \textsc{wmCPP}, a
\texttt{Window} is a unique identifier (e.g., a \rsin{usize}) for a window that,
again, maps windowing system specific representations to an abstraction that is
to be used within the window manager. A window in and of itself therefore does
not contain any \textit{state}. To associate state with a window, we wrap it in
a structure that we call a \textit{client}. Clients are objects that contain all
relevant information about a window: its title, position and size on-screen,
process identifier, whether it is in the fullscreen state, and much more.

\rssubsection{Mutable State with Cell and RefCell}

So far, we have deferred discussing a major aspect of the Rust programming
language: \textit{mutability}. Rust's \rsin{mut} keyword carries two kinds
of semantics, depending on where it is used. In patterns, \rsin{mut} indeed
indicates that mutating is allowed. In references, however, it rather pertains
to \textit{exclusivity}. That is, a \rsin{&mut} reference is not allowed to be
\textit{aliased}. Consider part of \textsc{wmRS}'s \rsin{Client} structure:

\begin{rustblock}
  pub struct Client {
    window: Window,
    name: RefCell<String>,
    parent: Option<Window>,
    children: RefCell<Vec<Window>>,
    fullscreen: Cell<bool>,
    managed_since: SystemTime,
    // ...
  }
\end{rustblock}

We want to pass around references to instances of this structure throughout
the \rsin{Model}, to read and alter state, and make window management
decisions accordingly. One possibility would be to have the different
client-mutating methods that are part of the \rsin{Model} take as parameter
a \rsin{&mut Client}. While this is a plausible approach, it heavily limits
us in using and passing around references. In concreto, no more than a single
reference to a to-be-changed \rsin{Client} may exist at a time. This is
particularly troublesome, as we need to store \rsin{Client} instances within the
\rsin{Model}, as follows:

\begin{rustblock}
  pub struct Model<'model> {
    client_map: HashMap<Window, Client>,
    // ...
  }
\end{rustblock}

Since mutability transitively affects all surrounding structures, mutating a
single field within a \rsin{Client} instance means that that instance would have
to be \rsin{&mut}. This additionally requires that the \rsin{Model}'s method
that retrieves and operates on such an instance would \textit{also} have to
take \rsin{&mut self}. As a result, something like the following would not be
possible:

\begin{rustblock}
  impl<'model> Model<'model> {
    fn set_fullscreen_window(&mut self, win: Window) {
      if let Some(c) = self.client_map.get_mut(&win) {
        self.set_fullscreen_client(c);
      }
    }
\end{rustblock}
\begin{rustblock}
    fn set_fullscreen_client(&self, cli: &mut Client) {
      cli.set_fullscreen(true);
    }
    // ...
  }
\end{rustblock}

Even though \rsin{set_fullscreen_client} can be called on a non-exclusive
reference to the model (i.e. it takes \rsin{&self}), the above will not compile,
as an immutable borrow of \rsin{self} occurs after a mutable one inside
\rsin{set_fullscreen_window}, which is not allowed due to the exclusivity rules
discussed above.

Moreover, it is rarely the
case that more than a single field of a \rsin{Client} instance is mutated.
The \rsin{set_fullscreen_client} method will only change the value of the
\rsin{fullscreen} field, for instance.

Furthermore, two types of mutability exist: \textit{interior}
and \textit{exterior} mutability. Part of \textsc{wmRS}'s client structure looks
as follows:

Each field that is expected to \textit{mutate} throughout program
execution---representing state changes instigated by windowing system
events---gets wrapped inside either a \rsin{Cell} or a \rsin{RefCell}.

That is,
an \textit{alias} of that instance may not exist elsewhere within the
program.  What \rsin{Cell} and \rsin{RefCell} allow
us to do, is to instead take as parameter a non-exclusive \rsin{&Client} as
parameter, and mutate individual fields, therein enabling reference aliasing.
Consider the following \rsin{Client} method:

\begin{rustblock}
  impl Client {
    pub fn set_fullscreen(&self, bool: value) {
      self.fullscreen.set(value);
    }
    // ...
  }
\end{rustblock}

To be able to call this method on a \rsin{Client} reference, that reference
does not need to be exclusive (due to it operating on \rsin{&self} instead of
\rsin{&mut self}).

% https://stackoverflow.com/questions/63487359/interior-mutability-abuse-in-api-design#:~:text=Conversely%2C%20C%2B%2B%20has%20nothing%20like,implicitly%20behind%20an%20UnsafeCell%20already.
