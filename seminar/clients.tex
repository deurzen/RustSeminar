%&latex
\section{Clients}
% TODO: - main data structure that represents a window under window manager
%         control
%       - wrapper around X11 window
%       - directly manipulated by practically all parts of the program
%       - reason for pointer (C++) vs id (Rust) representation

So far, the notion of a \textit{window} was used throughout without defining
what exactly it is to entail. In both \textsc{wmRS} and \textsc{wmCPP}, the
\texttt{Window} type is a unique identifier (e.g., an unsigned integer) for a
window that, again, serves to map windowing system specific representations to
an abstraction that is to be used within the window manager. A window in and of
itself therefore does not contain any \textit{state}. To associate state with a
window, we wrap around it a structure that we call a \textit{client}. Clients
are objects that contain all relevant information about a window with regard to
the window manager: its title, position and size on-screen, process identifier,
whether it is in fullscreen mode, and much more.

\rssubsection{Mutable State with Cell and RefCell}

We have thus far deferred discussing a major aspect of the Rust
programming language: \textit{mutability}. Rust's \rsin{mut} keyword
carries two kinds of semantics, depending on where it is used. In
patterns, \rsin{mut} indeed indicates that changing the underlying value
is allowed\cite{therustbook, intmutpatterns}. In references, however,
it rather pertains to \textit{exclusivity}. That is, a \rsin{&mut}
reference may be mutated, but additionally is not allowed to be
\textit{aliased}\cite{therustbook, intmutpatterns}. The reason this rule
exist, is because Rust's safety foundation is based on the principle
\text{\textit{aliasing} \texttt{XOR} \textit{mutability}}: An object can only be
safely mutated if no other reference exists to its contents\cite{therustbook,
therustreference}. Consider part of \textsc{wmRS}'s \rsin{Client} structure:

\begin{rustblock}
  pub struct Client {
    window: Window,
    workspace: Cell<usize>,
    parent: Option<Window>,
    children: RefCell<Vec<Window>>,
    fullscreen: Cell<bool>,
    // ...
  }
\end{rustblock}

We want to pass around references to instances of this structure throughout the
\rsin{Model}, to read and alter state, and to make window management decisions
accordingly. Disregarding the concrete contents of the \rsin{Client} structure
for a moment, one possibility would be to have the different client-mutating
methods that are part of the \rsin{Model} operate on a \rsin{&mut Client}.
While this is a plausible approach, it heavily limits us in using and passing
around references as, in concreto, no more than a single reference to a
to-be-mutated \rsin{Client} may then exist at a time\cite{therustbook}. This
becomes particularly troublesome, as we need to store \rsin{Client} instances
within the \rsin{Model}, mapping window representations to their associated
\rsin{Client} structure, as follows:

\begin{rustblock}
  pub struct Model<'model> {
    client_map: HashMap<Window, Client>,
    // ...
  }
\end{rustblock}

Since mutability transitively affects all surrounding structures, mutating a
single field within a \rsin{Client} instance means that the reference to that
instance would have to be \rsin{&mut}. This additionally requires that the
\rsin{Model}'s method that retrieves from its \rsin{client_map} and operates on
such an instance would \textit{also} have to take \rsin{&mut self}. For this
reason, this kind of mutability is what's known in Rust as \textit{inherited}
mutability, also referred to as \textit{external} mutability\cite{therustbook}.

Considering that the windowing system generates events in respect to some
\textit{window}, whenever we need to alter the state that is associated with
that window, we must therefore first retrieve a mutable reference to its client
from the \rsin{client_map}. A result of this, is that something like the
following would not be possible:

\begin{rustblock}
  impl<'model> Model<'model> {
    fn set_fullscreen_window(&mut self, win: Window) {
      if let Some(c) = self.client_map.get_mut(&win) {
        c.set_fullscreen(true);
        self.apply_layout(c.workspace());
      }
    }
\end{rustblock}
\begin{rustblock}
    fn apply_layout(&self, index: usize) { /* ... */ }
    // ...
  }
\end{rustblock}

Even though \rsin{apply_layout} can be called on a non-exclusive
reference to the model (i.e. it takes \rsin{&self}), the
above will not compile: An immutable borrow of \rsin{self}
(\rsin{|\bfseries{self}|.apply_layout(c.workspace())}) occurs \textit{after}
a mutable one (\rsin{|\bfseries{self}|.client_map.get_mut(&win)}) inside
\rsin{set_fullscreen_window}, which is not allowed due to the exclusivity rules
discussed before. Because every method that directly manipulates a client has a
structure analogous to the \rsin{set_fullscreen_window} method, this approach is
not viable.

To solve this, we use a technique called \textit{interior mutability}. Interior
mutability enables the mutating of an underlying structure of data with a
\textit{non-exclusive} reference to it, therein defying the requirement that
only exclusive references may be mutated\cite{therustbook}. To facilitate
this, four language constructs exist: \rsin{Cell} and \rsin{RefCell} for
single-threaded solutions, and \rsin{Mutex} and \rsin{RwLock} as their
thread-safe alternatives\cite{therustbook, therustreference}.

In view of the fact that window managers operate on an \textit{ordered}
stream of events, they function well as event-driven, single-threaded
applications. To that end, each \rsin{Client} field that is expected to
mutate throughout program execution---representing state changes instigated
by windowing system events---gets wrapped inside either a \rsin{Cell} or a
\rsin{RefCell}. \rsin{Cell} deals solely with \textit{values}: values are moved
or copied into the cell, or retrieved from it as a whole\cite{therustbook,
intmutpatterns}. Retrieving a reference to the contents of a \rsin{Cell} is
not possible\cite{therustbook, intmutpatterns}. \rsin{RefCell}, on the other
hand, works exclusively with \textit{references}, and implements so-called
\textit{dynamic borrowing}\cite{therustbook, intmutpatterns}. That means
that access to mutable references to the contents of a \rsin{RefCell} are
tracked \textit{at runtime}, which will cause a panic if a second mutable
borrow is attempted while another is already outstanding\cite{therustbook,
intmutpatterns}. Compared to \textit{static borrowing}, at a slight runtime
cost, dynamic borrowing permits much more flexible use of references, since
the compiler is only able to perform borrow checks at a rather coarse level of
granularity\cite{therustbook, intmutpatterns}.

From a practical point of view, what \rsin{Cell} and \rsin{RefCell} allow us to
do, is to instead take a non-exclusive \rsin{&Client} as parameter, and mutate
individual fields, therein enabling reference aliasing \textit{and} retaining
mutability. In a sense, interior mutability makes it possible to \textit{divide}
mutability of an entire structure into its individual constituents. Consider the
following \rsin{Client} method:

\begin{rustblock}
  impl Client {
    pub fn set_fullscreen(&self, bool: value) {
      // copy `value` into the Cell
      self.fullscreen.set(value);
    }
    // ...
  }
\end{rustblock}

Thanks to interior mutability, to be able to call this method on a \rsin{Client}
reference, that reference does not need to be exclusive (due to it operating
on \rsin{&self} instead of \rsin{&mut self}). This allows us to rewrite the
\rsin{set_fullscreen_window} method as follows:

\begin{rustblock}
  fn set_fullscreen_window(&self, win: Window) {
    if let Some(c) = self.client_map.get(&win) {
      c.set_fullscreen(true);
      self.apply_layout(c.workspace());
    }
  }
\end{rustblock}

Using this technique, \textit{most} \rsin{Model} methods will now take
\rsin{&self}, therein greatly improving flexibility, and consequently allowing
for a much better structuring of code.

\cppsubsection{Implicit interior mutability with references}

Disregarding the non-standardized \cppin{restrict} keyword, in \cpp,
\textit{every} reference (and pointer) can be aliased\cite{cppstd}. Moreover,
there is no inherent restriction on the mutability of references to an object.
Compared to Rust, all mutability in \cpp would therefore be considered interior
mutability. Part of \textsc{wmCPP}'s \cppin{Client} structure looks as follows:

\begin{cppblock}
  typedef struct Client* Client_ptr;
  typedef struct Client final
  {
    Window window;
    unsigned workspace;
    Client_ptr parent;
    std::vector<Client_ptr> children;
    bool fullscreen;
    // ...
  }* Client_ptr;
\end{cppblock}

Instead of encoding relationships with other client objects (i.e. \cppin{parent}
and \cppin{children}) using window representations (\cppin{Window}), we store
\textit{pointers} to the associated objects directly. Furthermore, we do not
need to make use of smart pointers (e.g. \cppin{std::shared_ptr}). The reason
we can safely store \textit{raw} pointers to other \cppin{Client}s within a
client structure, is thanks to a main event loop invariant we rely on: For each
window that gets destroyed, an event will be generated by the windowing system,
and propagated up to the window manager. That is, we know \textit{exactly} when
to deallocate a \cppin{Client} and all of its \cppin{children}. To map window
representations to their associated client structures, we again make use of a
\cppin{std::unoredered_map}:

\begin{cppblock}
  class Model final
  {
    // ...
  private:
    std::unordered_map<Window, Client_ptr> client_map;
    // ...
  };
\end{cppblock}

Each \cppin{Client_ptr} in the map is a heap-allocated object. To change the
state of a window, the \cppin{Model} has associated methods that retrieve and
mutate client objects, an example of which is the following:

\begin{cppblock}
  void
  Model::set_fullscreen_window(Window window)
  {
    if (client_map.find(window) != client_map.end()) {
      Client_ptr client = client_map.at(window);
      client->fullscreen = true;
      apply_layout(client->workspace);
    }
  }
\end{cppblock}

Most client-mutating methods that are part of the \cppin{Model} follow the same
approach.

\subsection{Clients: Remarks}

While Rust's mutability rules make the language much more secure in regard to
memory-safety and data races, its borrowing system is actually more restrictive
than strictly necessary to ensure these guarantees. That is to say, in a
\textit{perfect implementation}, static (i.e. compile-time) borrow checking
would be as flexible as dynamic (i.e. runtime) borrow checking. That is
consequently the main reason for the existence of Rust's interior mutability
constructs: This perfect implementation does not exist. And, for several good
reasons, two of which are language simplicity and compilation speed.

From a pure pragmatics standpoint, Rust's mutability restrictions add
significant additional accounting effort. In fact, they even render the
implementation of some programming patterns impossible. For instance, compared
to our \cpp examples, self-referential structures are not allowed in Rust.
For this reason, the only way to encode relationships to other instances is
through use of some kind of non-referential indirection. For our \rsin{Client}
structure in particular, that means we need to store the window representations
of \rsin{parent} and \rsin{children} instead of direct references to their
\rsin{Client} instances. This adds both a programming and runtime overhead, as
we repeatedly need to retrieve their instances through the \rsin{client_map}
whenever we need to read or alter their state.


% TODO: discuss problem of Rust mutability -> decourages encapsulation: even though 1 field needs to change, reference to _whole_ structure needs to be mutable

% https://stackoverflow.com/questions/63487359/interior-mutability-abuse-in-api-design#:~:text=Conversely%2C%20C%2B%2B%20has%20nothing%20like,implicitly%20behind%20an%20UnsafeCell%20already.
