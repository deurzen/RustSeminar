%&latex
\section{External Dependency Management}

As a programming  language's ability to aid the programmer  in managing external
dependencies---by,  for instance,  providing various  tools that  come installed
with  its compiler  or development  environment---is generally  not incorporated
into that language's syntax or, by extension, its semantics, it is traditionally
not considered an aspect of that language's feature set per se. Notwithstanding,
it is more  and more becoming an appreciated addition  to the \textit{ecosystem}
around a  language, especially so  for compiled  languages. In fact,  many would
consider automated  external dependency management to  be a must for  any modern
programming language.  As it directly affects  both the portability of  code, as
well as the (ease of) managing  different versions of a dependency, a language's
ability  to unburden  the  programmer  from the  manual  management of  external
dependencies  can greatly  improve the  maintainability  of a  project, and  can
therefore  indeed be  viewed  as a  feature  of  the language  in  light of  its
\textit{pragmatics}. In this  section, we will be  discussing the practicalities
of working with external code in both  Rust and C\texttt{++}. We will do this by
means of a comparison between two window manager implementations, one written in
Rust,  and the  other  written  in C\texttt{++},  which  we  will henceforth  be
referring to as \textsc{wmRS} and \textsc{wmCPP}, respectively.

Both  window managers  are built  on top  of the  X Window  System, which  means
they  rely on  an external  library  to interface  with the  X server.  Although
\textsc{wmRS} and  \textsc{wmCPP} each use  a different library to  achieve this
(XCB  over \texttt{libxcb}  and  Xlib over  \texttt{libX11}, respectively),  the
concept is the same,  as they both require the importing of  an external body of
code.

\subsection{Portability}
% TODO: - package manager(s), manully by developer (C++) vs Cargo (Rust)
%       - discrepancies between operating systems, platforms

C\texttt{++} does not come with an  external dependency manager. That is to say,
the  official  ISO  standard C\texttt{++}  specification\cite{cppstd}  does  not
outline the  functionality or otherwise existence  of any package manager.  As a
result, developers have  no other choice than to resort  to third-party tools to
manage  dependencies,  to  reinvent  the  wheel  themselves  and  hack  together
configure and build scripts that take  external dependencies into account, or to
disregard package management  altogether and let other developers  and end users
resolve  dependencies  on  their  own.  In  any  case,  C\texttt{++}  dependency
management is complex, and the lack of a standardized tool inhibits the adoption
of C\texttt{++}  projects, as well as  their portability to other  platforms and
development environments.

Since   the  only   external  dependency   \textsc{wmCPP}  is   reliant  on   is
\texttt{libX11},  which  is  fairly  ubiquitous and  readily  available  through
distributions'   own  package   managers,   it  merely   links   with  the   X11
development  libraries   without  the  use   of  any  form  of   own  dependency
management.  Specifically, building  the  project is  done  using the  following
\textbf{\texttt{Make}} script (large parts altered or redacted for clarity).

\begin{minted}{make}
  CC = g++
  CXXFLAGS = -std=c++17 -march=native -O3
  LDFLAGS = `pkg-config --libs x11` -flto

  obj/%.o: src/%.cc
    ${CC} ${CXXFLAGS} -MMD -c $< -o $@
  release: obj/%.o
    ${CC} $< ${LDFLAGS} -o bin/wmCPP
\end{minted}

Assuring the availability of an external body  of code is not the only aspect of
external  dependecy  management that  affects  the  pragmatics of  C\texttt{++}.
Writing a library that is to be imported as an external dependency and including
code from an external  dependency both require special care to  be taken to make
sure no symbol collisions occur. In header files, this is usually done by making
use of so-called  \textit{header guards}, a set of  preprocessor directives that
render including a  header an idempotent operation,  preventing double inclusion
errors. If incorrect double inclusion  does unexpectedly occur, for instance due
to  collisions in  the header  guards themselves,  vague---as header  guards are
handled  by  the preprocessor,  not  the  compiler---and hard-to-trace  compiler
errors  arise,  subjecting the  programmer  to  unnecessary mental  strain,  and
wasting  time. Although  not part  of the  official standard,  many C\texttt{++}
compilers  support the  \mintinline{cpp}{#pragma once}  preprocessor  directive,
providing the same functionality as header guards, but preventing the occurrence
of double inclusion errors.

Another source of issues is the order of include directives in C\texttt{++}. The
ordering of included  files relative to one another  can significantly influence
the outcome of the compilation step, possibly causing errors, one of which being
the introduction of circular dependencies  while there actually aren't any there
(e.g. a forward declaration that was included only after the circular dependency
was already introduced).\\

The Rust programming  language has its own package manager,  called Cargo. Cargo
can automatically download a project's external dependencies (and, transatively,
\textit{their} dependencies), compile them, and  install them locally, such that
they can be used  during the linking stage of the build  process. To make proper
use of Cargo within a Rust project, that project must be turned into a so-called
\textit{package}.  Rust  packages  are  simply  a  collection  of  source  files
along with  a manifest  file (named \texttt{Cargo.toml})  in the  project's root
directory. This manifest file describes  the package's meta-information (such as
its name  and version),  and a  set of  \textit{target crates}.  A crate  is the
source code or  compiled artifact of either a library  or executable program, or
possibly a  compressed package that is  grabbed from a registry  (a service that
provides  a  collection  of  downloadable crates).  A  package's  manifest  file
describes each of its target crates  by specifying their type (binary executable
or library), their metadata, and how to build them.

The  following manifest  file describes  the  package that  represents our  Rust
window manager, \textsc{wmRS} (parts redacted for clarity).\\

\begin{minted}{shell}
  [package]
  name = "wmRS"
  version = "0.1.0"
  authors = ["deurzen <m.deurzen@tum.de>"]
  edition = "2018"
  license = "BSD3"
  documentation = "https://docs.rs/wmRS"
  readme = "README.md"
  default-run = "wmRS"
  description = """
  An ICCCM & EWMH compliant X11 reparenting,
  tiling window manager, written in Rust
  """
  
  [profile.release]
  lto = true
  
  [lib]
  name = "winsys"
  path = "src/winsys/mod.rs"
  
  [[bin]]
  name = "wmRS"
  path = "src/core/main.rs"
  
  [[bin]]
  name = "wmRSbar"
  path = "src/bar/main.rs"
  required-features = ["bar"]
  
  [[bin]]
  name = "wmRSclient"
  path = "src/client/main.rs"
  required-features = ["client"]
  
  [features]
  bar = []
  client = []
  
  [dependencies]
  x11rb = {
    version = "0.8.0",
    features = [
      "cursor",
      "xinerama",
      "randr",
      "res"
    ]
  }
  strum = {
    version = "0.19",
    features = ["derive"]
  }
  strum_macros = "0.19"
  anyhow = "1.0.33"
  log = "0.4"
  simplelog = "0.8.0"
  nix = "0.19.0"
\end{minted}

Our package consists  of a single library, along with  three binary executables.
The library  is an abstraction above  and wrapper around the  interface with the
underlying windowing system. It defines a  single Rust trait that represents the
connection between the  window manager and the windowing  system. This decouples
the implementation  of the window  manager from that  of the interface  with the
windowing system, and  additionally allows for the seamless  transition from one
windowing system  to another, as multiple  windowing systems can be  targeted by
implementing the  trait, effectively creating  a new wrapper around  an external
library  that directly  interfaces that  windowing system.  Currently, only  the
interface with  the X  Window System  is implemented,  but interfacing  with the
newer and more modern Wayland is as easy as implementing a new connection to it.
A  small portion  of \textsc{wmRS}'s  winsys library  connection trait  looks as
follows.

\begin{minted}{rust}
  pub trait Connection {
    fn flush(&self) -> bool;
    fn step(&self) -> Option<Event>;
    fn connected_outputs(&self) -> Vec<Screen>;
    fn top_level_windows(&self) -> Vec<Window>;
    fn get_pointer_position(&self) -> Pos;
    fn warp_pointer(
      &self,
      pos: Pos,
    );
  
    fn init_window(
      &self,
      window: Window,
      focus_follows_mouse: bool,
    );
    fn init_frame(
      &self,
      window: Window,
      focus_follows_mouse: bool,
    );
    fn init_unmanaged(
      &self,
      window: Window,
    );
    fn cleanup_window(
      &self,
      window: Window,
    );
    fn map_window(
      &self,
      window: Window,
    );
    fn unmap_window(
      &self,
      window: Window,
    );
    fn reparent_window(
      &self,
      window: Window,
      parent: Window,
      pos: Pos,
    );
    fn unparent_window(
      &self,
      window: Window,
      pos: Pos,
    );
    fn destroy_window(
      &self,
      window: Window,
    );
    fn close_window(
      &self,
      window: Window,
    ) -> bool;
    fn kill_window(
      &self,
      window: Window,
    ) -> bool;
    fn place_window(
      &self,
      window: Window,
      region: &Region,
    );
    fn move_window(
      &self,
      window: Window,
      pos: Pos,
    );
    fn resize_window(
      &self,
      window: Window,
      dim: Dim,
    );
    fn focus_window(
      &self,
      window: Window,
    );
    fn stack_window_above(
      &self,
      window: Window,
      sibling: Option<Window>,
    );

    // ...
  }
\end{minted}


The  three binary  executable  crates  represent the  window  manager, a  client
program  to communicate  with  the  window manager  (to  control various  window
management  activities,  such as  closing  the  currently focused  window,  from
scripts or the  command line), and a status bar  that displays information about
the state of the window manager, such as the currently activated workspace.

\subsection{Versioning}
% TODO: - manual (C++) vs automatic (Rust) versioning
{ \textcolor{gray}\blindtext }
