%&latex
\section{External Dependency Management}
% TODO: - Xlib (C++ impl.) and XCB (Rust impl.) dependencies
%       - wrapper {class,struct} (resp.) and decoupling from external
%         dependencies
%       - {lack of,} (resp.) ease of using external libraries

As a programming  language's ability to aid the programmer  in managing external
dependencies---by,  for instance,  providing various  tools that  come installed
with  its compiler  or development  environment---is generally  not incorporated
into that language's syntax or, by extension, its semantics, it is traditionally
not considered an aspect of that language's feature set per se. Notwithstanding,
it is more  and more becoming an appreciated addition  to the \textit{ecosystem}
around a  language, especially so  for compiled  languages. In fact,  many would
consider automated  external dependency management to  be a must for  any modern
programming language.  As it directly affects  both the portability of  code, as
well as the (ease of) managing  different versions of a dependency, a language's
ability  to unburden  the  programmer  from the  manual  management of  external
dependencies  can greatly  improve the  maintainability  of a  project, and  can
therefore  indeed be  viewed  as a  feature  of  the language  in  light of  its
\textit{pragmatics}. In this  section, we will be  discussing the practicalities
of working with external code in both  Rust and C\texttt{++}. We will do this by
means of a comparison between two window manager implementations, one written in
Rust,  and the  other  written  in C\texttt{++},  which  we  will henceforth  be
referring to as \textsc{wmRS} and \textsc{wmCPP}, respectively.

Both  window managers  are built  on top  of the  X Window  System, which  means
they  rely on  an external  library  to interface  with the  X server.  Although
\textsc{wmRS} and  \textsc{wmCPP} each use  a different library to  achieve this
(XCB  over \texttt{libxcb}  and  Xlib over  \texttt{libX11}, respectively),  the
concept is the same,  as they both require the importing of  an external body of
code.

\subsection{Portability}
% TODO: - package manager(s), manully by developer (C++) vs Cargo (Rust)
%       - discrepancies between operating systems, platforms
{ \textcolor{gray}\blindtext }

\subsection{Versioning}
% TODO: - manual (C++) vs automatic (Rust) versioning
{ \textcolor{gray}\blindtext }
